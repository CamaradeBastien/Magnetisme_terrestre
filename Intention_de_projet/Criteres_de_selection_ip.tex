\documentclass{standalone}
\usepackage[utf8]{inputenc}
%\usepackage[francais]{babel}
\usepackage[T1]{fontenc}
\usepackage{amsmath}
\usepackage{amsfonts}
\usepackage{amssymb}
\usepackage{graphicx}
\graphicspath{{/home/utilisateur1/Documents/Genie_physique/H2018/Projet_II/Magnetisme_projet}}%changer le numéro du devoir
\usepackage{standalone}
\usepackage{tikz}
\renewcommand{\thesection}{\arabic{section}}
\author{Bastien Gauthier-Soumis,\\
 Edward Halle-Hannan, ,\\
Massine Kadi, ,\\
Félix Pelletier, }
\begin{document}
Critères de sélection : Description des critères de sélection et de leurs importances relatives ainsi que des contraintes.
	
\begin{enumerate}
	\item [1.]
\textbf{	Contrainte d’argent :}
	
	Le budget alloué pour le projet est de 100\$. Il est important que n’importe qu’elle solution envisagée respecte cette contrainte qui se veut binaire. Toute solution qui dépasse ce montant se verra refusée.
	Critère : oui / non\\
	
	
	\item [2.]
	\textbf{Contrainte de temps :}
	
	Il est important que la solution choisie soit réalisable dans le temps imparti, c’est-à-dire environ 9 semaines après la remise du premier rapport (14 Février) mettant en valeurs le choix de la solution. Nous estimons alors un total de 45 heures disponibles pour la réalisation. 
	Critère : (Dépasse largement les 45 heures) 1 2 3 4 5 (En deçà des 45 heures)\\
	
	
	\item [3.]
	\textbf{Critère de complexité des composantes utilisées :}
	
	Il est important que les composantes utilisées soit d’une complexité acceptable pour notre bagage scientifique et ne deviennent pas des boîtes noires. La solution choisit devra contenir une majorité de composantes construites par l’équipe et/ou de composantes dont le fonctionnement n’échappe pas à la compréhension de chacun. Une solution utilisant des principes physiques simples, mais obligeant des acrobaties techniques sera préférable à des technologies avancées.
	Critère : (Utilisation d’en senseur de champ magnétique terrestre) 1 2 3 4 5 (Nous n’utilisons aucune composante complexe)\\
	
	
	\item [4.]
	\textbf{Critère de qualité de la réponse du système :}
	
	L’évaluation durera un maximum de 5 minutes, il est primordial que la solution choisie réponde complètement à cette contrainte de temps. Les solutions seront donc évaluées pour leur aptitude à fournir en 5 minutes les réponses, sur la direction et la magnitude du champ, les plus précises et exactes. 
	Critère : (la solution fournie une réponse inexacte et imprécise)1 2 3 4 5 ( la solution fournie une réponse exacte et précise)\\
	
	
	\item [5.]
	\textbf{Critère de simplicité de la prise de données :}
	
	Il est important que la solution choisie permette la prise en charge complète de la mesure de façon informatique, c’est-à-dire que la solution permette, avec le moins de manipulation possible, à un utilisateur d’acquérir les données. Les solutions seront jugées sur le nombre de manipulations nécessaires à l’obtention des données. 
	Critère : (l’obtention des données nécessite plus de 5 manipulations) 1 2 3 4 5 (L’utilisateur n’a qu’un seul bouton à actionner) 
	
	\item[6.] 
\textbf{	Originalité de la solution :}
La solution proposée devra être originale au possible, soit dans les phénomènes physiques qu'elle utilise soit dans la méthode d'obtention des données, le traitement ou tout autre partie du processus nécessaire. Il serait préférable que le projet nous permette de mettre en application notre ingéniosité et de nous éloigner le plus possible des protocoles préétablies.   
\end{enumerate}

\textbf{Un mot sur l'importance relative de chacun des critères}
 Il est primordial que toute solution réponde à la contrainte d'argent sinon elle ne sera pas viable. Ensuite, dans l'ordre: le critère de performance est le plus important, suivi du critère temporel, de l'originalité et de la simplicité d'utilisation. Les raisons poussant à cette hiérarchie sont que nous voulons, avant tout, que le système produise des réponses précises et exactes sinon il ne répondrait pas au but de sa conception, ensuite nous savons que toute bonne idée ne vaut rien si nous n'avons pas le temps nécessaire pour la mener à bien, nous considérons que l'originalité nous permettra de nous éloigner des solutions toutes faites et de vraiment mettre à l'épreuve notre ingéniosité. Finalement, nous avons considéré la simplicité comme un critère important, mais de second ordre puisque son accomplissement réside dans le temps que nous aurons pour la partie "optimisation". Il est bon de noter que nous tenterons de faire apparaître l'importance relative de chacun des critères dans la grille.






\end{document}
