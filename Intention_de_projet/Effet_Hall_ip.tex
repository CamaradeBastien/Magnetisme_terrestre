\documentclass{standalone}
\usepackage[utf8]{inputenc}
%\usepackage[francais]{babel}
\usepackage[T1]{fontenc}
\usepackage{amsmath}
\usepackage{amsfonts}
\usepackage{amssymb}
\usepackage{graphicx}
\graphicspath{{/home/utilisateur1/Documents/Genie_physique/H2018/Projet_II/Magnetisme_projet}}%changer le numéro du devoir
\usepackage{standalone}
\usepackage{tikz}
\renewcommand{\thesection}{\arabic{section}}
\author{Bastien Gauthier-Soumis,\\
 Edward Halle-Hannan, ,\\
Massine Kadi, ,\\
Félix Pelletier, }
\begin{document}
Afin de mesurer un champ magnétique, il est possible d'utiliser un système exploitant l'effet Hall. Celui-ci consiste à la formation d'un potentiel proportionnel à un champ magnétique entre les côtés d'un conducteur traversé par un courant électrique.
Considérons un volume d'un matériaux conducteur électrique uniforme et parcouru par un courant $I$ entre deux extrémités dans la direction $x$. Si le conducteur est immergé dans un champ magnétique $\vec{B}$, les porteurs de charges en mouvement ressentirons une force magnétique $\vec{F_B}$ telle que \footnote{Où $\vec{v_d}$ est la vitesse de dérive des porteurs de charges, ici des électrons de charge $e$, qui décrit le mouvement global de ceux-ci et peut être définie selon $\vec{v_d} = \frac{\vec{j}}{nq}=\frac{\vec{j}}{ne}$, avec $\vec{j}$ la densité de courant et $n$ la concentration volumique de porteur de charge (qui peut être obtenue en observant les caractéristique du conducteur). Dans le cas où le conducteur est uniforme et de section constante $A$, $\vec{j}=\frac{I}{A}\hat{x}$, on a $\vec{v_d} = \frac{I}{neA}\hat{x}$}.
\begin{equation*}
\vec{F_B} = \vec{v_d}\times \vec{B}= \frac{I}{neA}\hat{x}\times \vec{B}
\end{equation*}
Les porteurs de charges tendrons donc à se déplacer aussi dans une direction perpendiculaire au courant, provoquant une séparation des charges et l'accumulation d'un potentiel électrique entre les deux côtés du conducteur. L'état du système deviendra rapidement stationnaire, et alors la forces électrique $\vec{F_{E_H}}$ (produite par le champ créé par l'accumulation de charge) et magnétique $\vec{F_B}$ s'annuleront.
\begin{equation*}
\vec{F}=\vec{F_{E_H}}+\vec{F_B}=\vec{0} \;\;\;\Rightarrow \vec{0}=e\vec{E_H}+\frac{I}{neA}\hat{x}\times \vec{B} \;\;\;\Rightarrow 0=eE_{Hy}+\frac{I}{neA}B_z
\end{equation*}
On peut exprimer le potentiel  par effet Hall $V_H$ produit par $\vec{E}_{Hy}$ uniforme (par accumulation de charge) entre les deux côtés du conducteur de largeur $l$, $V_H = -lE_{Hy}$.
\begin{equation*}
\Rightarrow 0=e\frac{-V_H}{l}+\frac{I}{neA}B_z \;\;\;\Rightarrow V_H = \frac{lI}{neA}B_z
\end{equation*}
Or, $V_H$ peut être mesuré, permettant de déterminer $B_z$.
En combinant trois systèmes semblables orientés selon des axes orthogonaux, on peut donc déterminer $B_z$,$B_y$ et $B_x$, identifiant l'orientation et la norme du champ magnétique ambiant (selon l'hypothèse qu'il est constant sur l'ensemble du système composé). 

Il est aisé d'acquérir des mesures de $V_{Hi}$ avec la plupart des interfaces de mesure analogique électronique, et peu de traitement de donné est nécessaire pour en tirer $\vec{B}$ (il s'agit d'une relation de proportionnalité constante pour chaque composante). Il serait donc probablement réalisable d'automatiser complètement des mesures exploitant l'effet Hall, et l'utilisation d'un tel détecteur serait à la fois très simple, nécessitant possiblement uniquement la pression d'un bouton, et très rapide, permettant qu'une grande quantité de mesures puissent être effectuées et traitées afin d'obtenir un résultat moyen en quelques secondes.
 De plus, plusieurs circuits intégrés basiques mesurant un $V_H$ selon un axe lorsque alimentés avec une tension de référence existent et s'avèrent peu coûteux.
Cependant, en pratique la vitesse de dérive dans un conducteur standard (pas supraconducteur) est très petite, et l'effet Hall ne produit des potentiels relativement élevés que si $\vert\vec{B}\vert$ est très grand, ce qui n'est pas le cas pour le champ magnétique terrestre. Les systèmes intégré mesurant l'effet Hall aisément disponibles sont rarement sensible à moins de $1\times10^{-4} \textit{T}$, voir  $1\times10^{-3} \textit{T}$, ce qui est insuffisant pour mesurer le champ magnétique terrestre, d'une amplitude d'approximativement $5\times10^{-5} \textit{T}$. Par ailleurs, construire un capteur à effet Hall complet par nos propres moyens impliquent de trouver ou fabriquer un conducteur uniforme et de le caractériser précisément, et il est don plus qu'improbable de réaliser un capteur plus performant que ceux accessible sur le marché.
\end{document}