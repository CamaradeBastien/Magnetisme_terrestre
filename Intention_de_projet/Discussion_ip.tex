\documentclass{standalone}
\usepackage[utf8]{inputenc}
%\usepackage[francais]{babel}
\usepackage[T1]{fontenc}
\usepackage{amsmath}
\usepackage{amsfonts}
\usepackage{amssymb}
\usepackage{graphicx}
\graphicspath{{/home/utilisateur1/Documents/Genie_physique/H2018/Projet_II/Magnetisme_projet}}%changer le numéro du devoir
\usepackage{standalone}
\usepackage{tikz}
\renewcommand{\thesection}{\arabic{section}}
\author{Bastien Gauthier-Soumis,\\
 Edward Halle-Hannan, ,\\
Massine Kadi, ,\\
Félix Pelletier, }
\begin{document}
Le modèle employé pour décrire le fonctionnement d'un magnétomètre à saturation comporte certaines limites qui devront être prises en compte afin de guider la conception de l'instrument de mesure, le caractériser et analyser les données qui résulteront de son opération. Tout d'abord, la relation entre $\vert\vec{B}\vert=B$ et $\vert\vec{H}\vert=H$ utilisée est idéalisée et grandement approximative. On a considéré 
\begin{equation*}
\end{equation*}

\end{document}